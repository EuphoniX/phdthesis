%!TEX root = ../thesis.tex

\chapter{Conclusions}
\label{ch:conclu}

\ifpdf
	\graphicspath{{Chapter11/Figs/Raster/}{Chapter11/Figs/PDF/}{Chapter11/Figs/}}
\else
	\graphicspath{{Chapter11/Figs/Vector/}{Chapter11/Figs/}}
\fi

This chapter summarises the contributions of this thesis and presents possible outlines for future research directions. While this thesis explores problems across an interdisciplinary field related to intelligent vehicles, certain prevailing conclusions can be drawn in this respect.

\section{Overall Findings}
This thesis has resulted in the following contributions to the research in applied autonomous driving and electric vehicles. 

In Chapter~\ref{ch:cmrn}, a visual approach to multi-robot navigation was conceptualised. This method utilised an existing multi-robot system that was initially developed for distributed cooperative SLAM to solve localisation problems relating to wheel slip and obstacle detection, leading to the incorporation of visual odometry and semantic segmentation into the system. Evaluations have verified the feasibility of these algorithms in tangible outdoor environments, thereby motivating their implementations on frameworks for autonomous cars.

Chapters~\ref{ch:semseg} through~\ref{ch:sim} have described subsequent implementations of these visual navigation approaches with an emphasis on autonomous driving. This began with semantic segmentation validations on Perth roads in Chapter~\ref{ch:semseg}, which resulted in robust classification accuracies and frame rates that are adequate for practical autonomous drives, which is further enhanced with LiDAR measurements. This was then integrated first as a module in a C++-based autonomous driving framework in Chapter~\ref{ch:modular}, and then as part of an improved ROS-based framework in Chapter~\ref{ch:evo}. In both instances, the visual navigation algorithms were supplemented with additional sensors such as LiDARs and IMUs to establish a holistic driving system. Semantic segmentation results were successfully used for the detection of road regions and lane markers, enabling the vehicle to achieve lane keeping and scene understanding. Visual odometry was implemented as described Chapter~\ref{ch:evo}, which saw the optimisation of an existing method to exploit the architecture of embedded computers, resulting in real-time accurate localisations. In addition to road lanes, visual object detection and recognition was utilised to construct path delimiters for drive tests. In this case, traffic cones were classified through an SVM and their positions are accurately ascertained by fusing LiDAR measurements, thereby producing an open path with the cones placed at either side. This method was verified first on the real system in Chapter~\ref{ch:evo}, and then on a simulation platform in Chapter~\ref{ch:sim}. For both applications, the cone detection algorithms were able to detect and position cones in real-time, enabling the vehicle to autonomously navigate the path. 

Chapter~\ref{ch:review} introduced the research on electric vehicles in this thesis through the presentation of a cloud-based telemetry platform, REView for data collection and analyses. This platform was programmed as a hybrid V2C/I2C solution for connected EVs, charging infrastructures and energy sources that are capable of aggregating data and interpreting it in a cohesive and meaningful way. Processed data are then visualised as a series of tables and charts with gamification features to critically inform EV users and station operators, in addition to providing automated billings to support network monetisation. Each feature on REView was modularly programmed to ensure scalability and improve interoperability for upcoming infrastructures and technologies. Usage forecasts for charging infrastructures have predicted the increase in popularity of fast chargers as EVs are shipped with larger battery capacities, whereas AC charging may be phasing into obsolescence with its lower charging frequencies. Results originating from REView have since facilitated significant headways into the investigation of Western Australia's EV landscape. Chapter~\ref{ch:charging} is an example of this contribution; data from REView was used to compared infrastructure usages between AC and DC charging. The study began with a comprehensive overview of the current EV charging outlook and proceeds with a consecution of time series analyses across AC and DC charging, with comparisons drawn against data from the RAC Electric Highway. Analysis results have demonstrated that charging behaviours differ across different charging types and that the location and usage cost of charging infrastructures directly affects its popularity. Following this, a cost model is also presented to illustrate the cost of charging infrastructure ownership and concluded that faster chargers will better benefit from higher usage traffic which will quickly offset the higher initial cost of investment.

\section{Future Research Recommendations}
Throughout the compilation of this thesis, there had been limitations that were identified, thereby disseminating further research questions. Notable recommendations for future works are described in the following paragraphs at a higher level.

\paragraph{Improved optimisations on visual navigation algorithms} Much of the algorithmic implementations described in this thesis were performed on embedded computers which provide limited computation headroom when compared to a workstation computer. While preliminary optimisations were performed for visual odometry, semantic segmentation was not subjected to this treatment which can yield higher frame rates. Segmentations can be optimised either by segregating ROIs (such as the road region for lane following routines) or by ignoring classes that are less relevant to a routine. Further, these methods can leverage on the rapid enhancements of deep learning libraries such as TensorFlow (including the upcoming TensorFlow 3.0) to easily design platform-optimised architectures. Contributing to this ease are also the availabilities of newer embedded computers that are purpose-built for deep learning applications from Nvidia. Its CUDA parallel computing platform and cuDNN deep learning library are well-supported for their compute hardware, capable of accelerating deep learning performances including TensorFlow and Caffe.

\paragraph{End-to-end multi-sensor driving system} The autonomous driving frameworks that were described in this thesis uses mediated perception that sees the decoupling of environmental perception and decision making. On the contrary, an end-to-end approach applies machine learning methods directly onto the control system with inputs such as steering, braking and acceleration. Using an end-to-end model that incorporates accurate localisation and scene understanding will introduce a cohesive deep learning paradigm for autonomous driving.
Preliminary works to extend the software framework in Chapter~\ref{ch:evo} into an end-to-end solution is currently in the works.%is documented as a whitepaper in Appendix~\ref{ch:app1}. 
This approach utilises existing sensors and compute hardware along with the simulation system described in Chapter~\ref{ch:sim} to present a multi-sensor solution for autonomous road drives. %Sensor fusion

\paragraph{Electromobility penetration forecasts} The plethora of data collected throughout the years on REView presents myriad possible analytics beyond its usage forecasts. Using UWA's DC charger as an example, the trends observed for charge frequencies, duration and energy consumption can be reinterpreted to predict Perth's electric vehicle landscape, as it experiences a consistent usage frequency. For instance, models from charging frequencies and the number of EVs in Perth can be heterogeneously fused to improve prediction accuracies of EV numbers. Likewise, data relating to charging energy delivery can be used to predict battery capacities of future EVs. Predictions like these can therefore be used as critical information such as policy whitepapers and consumer education.   %Using available data from the infrastructures.

\paragraph{Cloud-based big data analytics} Throughout the preparation of works presented in this thesis, an understanding was established that much of the efforts, be it in autonomous driving or electromobility, are homogeneously converging towards a big data problem. Similar to the connected infrastructures, the autonomous driving software generates large amounts of complex real-time data from its interfaces whereby the use of big data analytics can greatly benefit and streamline any recursive learning routines. This can be further enhanced with cloud computing, taking advantage of high-speed, low-latency mobile connectivity to introduce a centralised, large-scale deployment of an edge-based multi-agent autonomous vehicle system over a V2V/V2C model. On the other hand, the nature of REView's current data structure can greatly benefit from a big data framework as data collection and visualisation occur in real-time. REView, when properly redesigned over a PaaS model for big data analytics, will thereby provide high-performance data acquisition with guaranteed scalability as a long-term solution to cater for the increasing number of EVs and charging infrastructures. When cloud-based autonomous driving is deployed alongside REView over a PaaS model, the foundation for a highly intelligent framework for smart cities can be established. This interconnectivity will enable an autonomous EV to access information relating to charging infrastructures such as its location and type, where it can further access information from the smart grid to schedule ideal charging patterns for its driving behaviours. Ultimately, this could thereby enable perpetual autonomous drives without user interference. 

%Autonomous driving, EV, data.

\section{Final Remarks}
This thesis has documented the application of visual navigation algorithms and the development of comprehensive software frameworks for autonomous driving and electric vehicles. Visual navigation algorithms were first verified for application feasibility, implemented first on mobile robots as a precursor, and then on purpose-built autonomous driving frameworks for real-world evaluations. Continuing on the trend of software framework developments is the presentation of an intelligent telemetry platform, REView, for electric vehicles and their infrastructures. Analytics stemming from REView have aided in the study of charging behaviours, usage forecasts and ownership costs for charging infrastructures across Western Australia.

As we observe this research field transition from an engineering to a machine learning approach, products that incorporate intelligent solutions for vehicles are becoming commonplace. Some of these proliferations evolve into proposals of distributed systems that encompass the field of connected mobility, often incorporating external infrastructures that interconnect through the IoV. As computers are inherently electronic devices, it easily interfaces with drive-by-wire vehicles; having an electric vehicle thereby consolidates the computer, control and battery management systems into a cohesive platform. This rationalises any decision to build a connected autonomous electric vehicle. In the case of our testbed, it achieves this through the simple addition of a cellular modem, allowing users to control its computer over the Internet. This implication is far-reaching, and that the efforts that this thesis describes are preliminary in comparison. Examples in addition to the said research recommendations would include a centralised, cloud-based autonomous system for distributed vehicles. This could, for example, benefit services such as ridesharing, which are operable by governments or private entities. 

More importantly, I have since observed that the algorithms for intelligent transportation are rapidly established through the availability of large open-source communities and libraries, and its often high computation requirements will eventually be resolved with the availability of more efficient computers. Conversely, the social acceptance into these technologies remains a salient aspect of building a sustainable economy for these products. The suggestion for big data as a research recommendation is persistent across all levels of study, with implications extending beyond engineering, legislation and the economy. For instance, local governments and communities are striving to enact policies and prepare for their inevitable advent, but often lack an understanding of how the technologies will affect them in the future. Unlike technological innovations that can blanket entire product lines on a global scale, their public acceptance and penetration are context-specific and spatial, and therefore any analysis will have to apply accordingly. The presence of large data output from the system and the availability of high-speed, low-latency wireless connectivity such as 5G networks are further encouraging this research. This confluence of technologies that encompasses connected intelligent transport, big data/cloud computing and 5G will catalyse the development of connected transport systems and the IoV, introducing a research gap that continuously attracts fast-paced developments. The untapped potentials that arise from these technologies will render any structured data to be immensely valuable.

The works described in this thesis originates within The REV Project and are testaments to its mission in developing intelligent and sustainable transportation solutions. We hope that the presentation of these readings is able to encourage future advancements toward this fast-paced research field.