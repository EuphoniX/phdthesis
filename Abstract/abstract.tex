% ************************** Thesis Abstract *****************************
% Use `abstract' as an option in the document class to print only the titlepage and the abstract.
\begin{abstract}
The growing ubiquity of electric vehicles is often characterised through its increasing autonomy and connectivity. This has led to catalyse the foundations of smart cities and intelligent transportation systems, where the applications of electromobility is often given a pivotal role towards their realisation. 

As the title suggests, this thesis presents its investigations into electromobility applications across two key areas --- (1) computer vision-based autonomous driving, and (2) data management and analyses of electric vehicle charging stations. 

The study into vision-based navigation aims to address the problem of developing an autonomous driving system that predominantly utilises the camera as the vehicle’s primary environmental perception sensor. This research gap is attributed to the greater algorithmic complexity in computer vision, as compared to LiDARs or radars. Additionally, the general attainability of cameras, and the diminishing cost of parallel computation has further contributed towards the motivation for this study. To this end, the requirements for visual navigation are centred upon localisation and scene understanding. More specifically, this thesis describes applications pertaining to visual odometry and semantic segmentation following an extensive literature survey. These methods are first tested for its feasibility on datasets and mobile robots, and then verified on an autonomous Formula-SAE electric car as the test bed, enabling the vehicle to perform object detection, lane keeping and dead reckoning in real-time. Experiments were conducted for road scenes and traffic cone drives, yielding fast and accurate results for detections, classifications and odometry.

The electric vehicle charging station network managed by The REV Project comprise of 23 7 kW AC stations and a 50 kW DC fast charging station. Each station is connected to a centralised server over the mobile network, perpetually transmitting telemetric data to the server’s daemons. The data generated from these stations effectuates the investigation into the charging behaviours across AC and DC stations, leading to the study of electric vehicle trends around Perth. Results from this study comprise of a combination of time series analyses that compares the cycles and energy consumption between AC and DC charges among local stations. A web-based telemetry monitoring platform, REView, is further described in this thesis. In addition to the charging stations, REView consolidates data from the project’s electric vehicle fleet and solar power generation into a unified framework that features on-demand monitoring for connected infrastructures. These are further detailed according to its back-end processes, encompassing its communication architectures, data management, data visualisation and presentation. 

The cumulation of works that are presented here conforms to The REV Project’s goal that describe contributions towards the fields of intelligent vehicles and the Internet of Vehicles. These contributions are exemplified in this thesis through the successful application of visual autonomous driving, and the analyses towards the electric vehicle trends in Perth, which should subsequently encourage further developments in this area.


\end{abstract}
